\documentclass[12pt,a4paper]{scrartcl}
\usepackage[utf8]{inputenc}
\usepackage[T1]{fontenc}
\usepackage[french]{babel}
\usepackage{textcomp}
\usepackage{amsmath,amssymb}
\usepackage{lmodern}
\usepackage{graphicx}
\usepackage[dvipsnames,svgnames]{xcolor}
\usepackage{microtype}
\usepackage{hyperref}
\usepackage{lipsum}
\usepackage[pointedenum]{paralist}
\usepackage{listings}

\lstset{
language=python,
backgroundcolor=\color{Navy!10},
keywordstyle=\color{ForestGreen}\bfseries,
% numbers=left, numberstyle=\tiny,
tabsize=4,
}

\hypersetup{colorlinks=true,linkcolor=Brown,breaklinks=true,pdfstartview=XYZ}
\frenchsetup{StandardItemLabels=true}

\title{Équations différentielles}
\author{Gloria Faccanoni}
\date{15 janvier 2014}

\begin{document}
\maketitle

\begin{abstract}
\lipsum[1][1-5]
\end{abstract}

\tableofcontents

\addsec{Introduction}
\lipsum[1-2]

\section{Rappels}
\label{eti}
\lipsum[3]

\subsection{Condition Initiale}
\lipsum[3]

\subsection{Problème de Cauchy}
\lipsum[4]

\section{Exercices}
\lipsum[7]

\section{Suite...}
On a vu dans la section~\ref{eti} page~\pageref{eti} les rappels nécessaire pour continuer la leçon.

\footnote
{A apprendre car vous êtes trop nuls}

\section{Entrainement sur les listes}
\begin{enumerate}
\item Ont des ailes
	\begin{enumerate}
	\item Volent
		\begin{enumerate}
		\item Hirondelle
		\item Aigle
		\item Ara
		\end{enumerate}
	\item Ne volent pas
		\begin{enumerate}
		\item Poule
		\item Autruche
		\end{enumerate}
	\end{enumerate}
	
\item Vivent dans la mer
	\begin{itemize}
	\item Respirent dans l'eau
		\begin{itemize}
		\item requin
		\item hippocampe
		\end{itemize}
	\item Respirent hors de l'eau
		\begin{itemize}
		\item Baleine
		\item Dauphin
		\end{itemize}
	\end{itemize}
	
\item Vivent sur terre
	\begin{description}
	\item[Australie:] kangourou, koala
	\item[Europe:] loup
	\end{description}	 
\end{enumerate}

\section{Code}

\lstinputlisting{src.py}
\end{document}
