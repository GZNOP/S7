\documentclass[12pt,a4paper]{scrartcl}
\usepackage[utf8]{inputenc}
\usepackage[T1]{fontenc}
\usepackage[french]{babel}
\usepackage{textcomp}
\usepackage{amsmath,amssymb}
\usepackage{lmodern}
\usepackage{graphicx}
\usepackage[dvipsnames,svgnames]{xcolor}
\usepackage{microtype}
\usepackage{hyperref}
\usepackage{lipsum}
\usepackage[pointedenum]{paralist}
\usepackage{listings}
\usepackage{graphicx}
\usepackage{multicol}
\usepackage{xspace}
\usepackage{xcolor}
\usepackage{amsthm}
\usepackage{tikz}


\hypersetup{colorlinks=true,linkcolor=Brown,breaklinks=true,pdfstartview=XYZ}
\frenchsetup{StandardItemLabels=true}

\title{La Théorie des graphes}
\author{Fabio Daussy}
\date{\today}

\theoremstyle{plain}
	\newtheorem{theoreme}{Théorème}[section]
	\newtheorem{proposition}[theoreme]{Proposition}
	\newtheorem{corollaire}[theoreme]{Corollaire}
	\newtheorem{definition}[theoreme]{Définition}
	\newtheorem{exo}{Exercice}

\theoremstyle{remark}
	\newtheorem*{exemple}{Exemple}

\begin{document}
\maketitle

\begin{abstract}
Dans ce cours texte, nous allons voir brievement des éléments essentiels de la théorie des graphes. Inspiré du cours de Philippe Langevin.
Notamment sa définition et jusqu'où le sujet peut s'étendre.
\end{abstract}

\tableofcontents


\section{Notion de bases}

\begin{definition}
Un graphe $G$ est un couple $(X,U)$, où $X$ est un ensemble de \textbf{sommets} et
$U$ un sous ensemble tel que $U \subseteq \mathcal{P}_2(X)$, ce sont les \textbf{arêtes} du graphe 
\end{definition}

\begin{exemple}
	Ainsi, pour le graphe maison G en figure \ref{maison} nous avons :
$X:= \{ 1,2,3,4,5 \}\quad
U:= \{ (1,2),(1,3),(3,4),(2,4),(2,5),(4,5),(2,3),(1,4) \}$	
\end{exemple}

\begin{definition}
	On dit que deux sommets $x,y \in X$ sont \textbf{adjacents} si et seulement si $(x,y) \in U$.
\end{definition}

\begin{definition}
	On appelle le \textbf{degré} d'un sommet le nombre de sommets auquels ce sommet est adjacent. le degré d'un sommet $s \in X$ se note aussi $deg(s)$
\end{definition}

\begin{exemple}
	L'ensemble des degrés des sommets $s \in X$ du graphe $G$ (\ref{maison})
	\begin{table}[]
	\begin{tabular}{|l|l|l|l|l|l|}
	\hline
	sommet & 1 & 2 & 3 & 4 & 5 \\
	\hline
	degré  & 3 & 4 & 3 & 4 & 2 \\
	\hline
	\end{tabular}
	\end{table}
\end{exemple}

\begin{proposition}
	Soit $m$ le nombre d'arêtes du graphe
	\begin{equation}
		\sum_{s \in X} \text{deg}(s)\quad=\quad 2\times m
	\end{equation}\label{}
		
\end{proposition}

\begin{definition}
	On appelle l'\textbf{ordre} d'un graphe $G$ le nombre de sommets qui le composent (i.e. le cardinal de $X$)
\end{definition}

\begin{exemple}
	L'ensemble des degrés des sommets de $G$ (\ref{maison})
\end{exemple}

\begin{definition}
	
\end{definition}



\begin{figure} % flottant
	\centering
	\begin{tikzpicture}[scale=.9,auto=center,every node/.style={circle,fill=blue!		20}]
  	\node (a1) at (0,0) {1};
  	\node (a2) at (0,4) {2};
  	\node (a3) at (4,0) {3};
  	\node (a4) at (4,4) {4};
  	\node (a5) at (2,6) {5};
  
  	\draw (a1) -- (a2);
  	\draw (a1) -- (a3);
  	\draw (a2) -- (a4);
  	\draw (a4) -- (a3);
  	\draw (a2) -- (a5);
  	\draw (a4) -- (a5);
  	\draw (a2) -- (a3);
  	\draw (a4) -- (a1);
	\end{tikzpicture}
	\caption{Exemple : Le graphe maison $G(X,U)$ }\label{maison}
\end{figure}


\section{Connexité}

\section{Graphe Eulerien}

\end{document}