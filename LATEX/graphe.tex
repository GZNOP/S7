\documentclass[12pt,a4paper]{scrartcl}
\usepackage[utf8]{inputenc}
\usepackage[T1]{fontenc}
\usepackage[french]{babel}
\usepackage{textcomp}
\usepackage{amsmath,amssymb}
\usepackage{lmodern}
\usepackage{graphicx}
\usepackage[dvipsnames,svgnames]{xcolor}
\usepackage{microtype}
\usepackage{hyperref}
\usepackage{lipsum}
\usepackage[pointedenum]{paralist}
\usepackage{listings}
\usepackage{graphicx}
\usepackage{multicol}
\usepackage{xspace}
\usepackage{xcolor}
\usepackage{amsthm}

\hypersetup{colorlinks=true,linkcolor=Brown,breaklinks=true,pdfstartview=XYZ}
\frenchsetup{StandardItemLabels=true}

\title{La Théorie des graphes}
\author{Fabio Daussy}
\date{\today}

\theoremstyle{plain}
	\newtheorem{theoreme}{Théorème}[section]
	\newtheorem{proposition}[theoreme]{Proposition}
	\newtheorem{corollaire}[theoreme]{Corollaire}
	\newtheorem{definition}[theoreme]{Définition}
	\newtheorem{exo}{Exercice}

\begin{document}
\maketitle

\begin{abstract}
Dans ce cours texte, nous allons voir brievement des éléments essentiels de la théorie des graphes. Inspiré du cours de Philippe Langevin.
Notamment sa définition et jusqu'où le sujet peut s'étendre.
\end{abstract}

\tableofcontents


\section{Qu'est ce qu'un graphe}

\begin{definition}
Un graphe $G$ est un couple $(X,U)$, où $X$ est un ensemble de sommets et
$U$ un sous ensemble tel que $U \in \mathcal{P}_2(X)$, ce sont les arêtes du graphe 
\end{definition}





\end{document}